\documentclass[10pt]{article}
\usepackage{geometry}
\geometry{a4paper,scale=0.99}
\usepackage{amsmath,amsthm,amssymb,amscd}
\usepackage{latexsym}
\usepackage{indentfirst}
\usepackage{subfigure}
\usepackage{graphicx}
\usepackage{extarrows}
\usepackage{bm}
\usepackage{multirow}
\usepackage{algorithm}
%\usepackage{algorithmic}
\usepackage{algpseudocode}
\usepackage{setspace}
\usepackage{stackengine}
\usepackage{braket}
\usepackage{mathtools}
\usepackage{physics}
%\usepackage{commath}
\parskip 1em
\usepackage{authblk}
\newtheorem{definition}{Definition}
\newtheorem{theorem}{Theorem}
\newtheorem{lemma}{Lemma}
\newtheorem{remark}{Remark}
\newtheorem{condition}{Condition}
\newtheorem{operation}{Operation}
\newtheorem{proposition}{Proposition}
\newtheorem{corollary}{Corollary}
\newtheorem{assumption}{Assumption}
\newtheorem{problem}{Problem}
\newtheorem{example}{Example}
\renewcommand{\algorithmicrequire}{\textbf{Input:}}
\renewcommand{\algorithmicensure}{\textbf{MCMC Input:}}
\usepackage{color,xcolor}
\usepackage{cancel}
\DeclareMathOperator*{\argmax}{arg\,max}
\DeclareMathOperator*{\argsup}{arg\,sup}

\labelformat{enumii}{\theenumi(#1)}
%\date{\today}



%\bibliographystyle{unsrt}
%\bibliographystyle{plain}
\linespread{2}


\begin{document}

\title{
  \bf Discrete diffuser for CRISPR.
}

\author[1]{Jingwei Li}
\affil[1]{Shanghai Center for Systems Biomedicine, Shanghai Jiao Tong University, Shanghai 200240, China, (Email:ljw2017@sjtu.edu.cn).}

\baselineskip=0pt


%\pacs{}
%
%\keywords{}

\maketitle

\begin{abstract}
Discrete diffuser for CRISPR.
\end{abstract}

\section{$q_{0|t}\qty(x^{(1:D)}_0\middle|x^{(1:D)}_t)$ is not decomposable for discrete-time discrete-space diffusion}
Because $q_0\qty(x^{(1:D)}_0)$ is not decomposable, $q_{0|t}\qty(x^{(1:D)}_0|x^{(1:D)}_t)$ is not decomposable as well. However, the parameterization
\begin{eqnarray}\label{Eqp0tCompose}
  p^\theta_{0|t}\qty(x_0^{(1:D)}|x_t^{(1:D)}) = \prod_{d=1}^D p^{\theta,(d)}_{0|t}\qty(x_0^{(d)}|x_t^{(1:D)})
\end{eqnarray}
is widely used to efficiently calculate (especially for large $D$)
\begin{eqnarray*}
  p^{\theta,(d)}_{s|t}\qty(x^{(d)}_s|x^{(1:D)}_t)\coloneq \sum_{x^{(d)}_0}q^{(d)}_{s|0,t}\qty(x^{(d)}_s|x^{(d)}_0,x^{(d)}_t)p^{\theta,(d)}_{0|t}\qty(x^{(d)}_0|x^{(1:D)}_t).
\end{eqnarray*}

In https://arxiv.org/pdf/2402.03701, the authors propose a special first-order (i.e. markov) process that the transition probability matrix is
\begin{eqnarray}\label{EqInterpolarQ}
  Q^{(d)}_t = \alpha^{(d)}_t I + \qty(1-\alpha^{(d)}_t)\mathbf{1m}_d^\top,
\end{eqnarray}
where $\mathbf{m}_d\ge 0$ and $\langle\mathbf{1}, \mathbf{m}_d\rangle=1$. Then
\begin{eqnarray*}
  &&\alpha^{(d)}_{t|s}\coloneq \prod_{i=s+1}^t \alpha^{(d)}_i,\\
  &&Q^{(d)}_{t|s}\coloneq Q^{(d)}_{s+1}Q^{(d)}_{s+2}Q^{(d)}_{s+3}\cdots Q^{(d)}_t = \alpha^{(d)}_{t|s} I + \qty(1-\alpha^{(d)}_{t|s})\mathbf{1m}_d^\top.
\end{eqnarray*} 
It is usually assumed that $\lim_{t\to \infty}\alpha^{(d)}_{t|0}=0$, so the first-order process gradually becomes a zero-order process that $x^{(d)}_t$ are approximately independent of $x^{(d)}_{t-1}$. Also,
\begin{eqnarray}\label{EqsOn0tq}
  &&q^{(d)}_{s|0,t}\qty(\cdot|x^{(d)}_0,x^{(d)}_t) = \frac{\qty(Q^{(d)}_{s|0})^\top \mathbf{x}^{(d)}_0\circ Q^{(d)}_{t|s}\mathbf{x}^{(d)}_t}{\qty(\mathbf{x}^{(d)}_0)^\top Q^{(d)}_{t|0}\mathbf{x}^{(d)}_t}\\
  &&=\left\{\begin{array}{ll}
    \qty(1-\lambda^{(d)}_{t|s})\mathbf{x}^{(d)}_t + \lambda^{(d)}_{t|s}\mathbf{m}_d, & x_0^{(d)}=x^{(d)}_t,\\
    \qty(1-\mu^{(d)}_{t|s})\mathbf{x}^{(d)}_0 + \mu^{(d)}_{t|s}\alpha^{(d)}_{t|s}\mathbf{x}^{(d)}_t + \mu^{(d)}_{t|s}\qty(1-\alpha^{(d)}_{t|s})\mathbf{m}_d, & x_0^{(d)}\neq x_t^{(d)},
  \end{array}\right.
\end{eqnarray}
where
\begin{eqnarray*}
  &&\lambda^{(d)}_{t|s}\coloneq\frac{\qty(1-\alpha^{(d)}_{s|0})\qty(1-\alpha^{(d)}_{t|s})\left\langle \mathbf{m}_d,\mathbf{x}^{(d)}_t\right\rangle}{\alpha^{(d)}_{t|0} + \qty(1-\alpha^{(d)}_{t|0})\left\langle \mathbf{m}_d,\mathbf{x}^{(d)}_t\right\rangle},\\
  &&\mu^{(d)}_{t|s}=\frac{1-\alpha^{(d)}_{s|0}}{1-\alpha^{(d)}_{t|0}}.
\end{eqnarray*}

Because
\begin{eqnarray*}
  &&p^\theta_{s|t}\qty(x_s^{(1:D)}\middle|x_t^{(1:D)}) = \sum_{x_0^{(1:D)}}\prod_{d=1}^D q_{s|0,t}^{(d)}\qty(x_s^{(d)}\middle|x_0^{(d)}, x_t^{(d)})p^\theta_{0|t}\qty(x_0^{(1:D)}\middle|x_t^{(1:D)})\\
  &&=\sum_{x_0^{(D)}}q_{s|0,t}^{(D)}\qty(x_s^{(D)}\middle|x_0^{(D)}, x_t^{(D)})\sum_{x_0^{(D-1)}}q_{s|0,t}^{(D-1)}\qty(x_s^{(D-1)}\middle|x_0^{(D-1)}, x_t^{(D-1)})\cdots\sum_{x_0^{(1)}}q_{s|0,t}^{(1)}\qty(x_s^{(1)}\middle|x_0^{(1)}, x_t^{(1)})p^\theta_{0|t}\qty(x_0^{(1:D)}\middle|x_t^{(1:D)}).
\end{eqnarray*}
Calulating
\begin{eqnarray*}
  \mathcal{Z}_1\qty(x_s^{(1)},x_0^{(2:D)})\coloneq\sum_{x_0^{(1)}}q_{s|0,t}^{(1)}\qty(x_s^{(1)}\middle|x_0^{(1)}, x_t^{(1)})p^\theta_{0|t}\qty(x_0^{(1:D)}\middle|x_t^{(1:D)})
\end{eqnarray*}
for all $x_s^{(1)}$ and $x_0^{(2:D)}$ has complexity $O\qty(S_1\prod_{d=1}^D S_d)$ ($S_d$ is the possible value number of dimension $d$). Calculating
\begin{eqnarray*}
  \mathcal{Z}_2\qty(x_s^{(1:2)},x_0^{(3:D)})\coloneq\sum_{x_0^{(2)}}q_{s|0,t}^{(2)}\qty(x_s^{(2)}\middle|x_0^{(2)}, x_t^{(2)})\mathcal{Z}_1\qty(x_s^{(1)},x_0^{(2:D)})
\end{eqnarray*}
for all $x_s^{(1:2)}$ and $x_0^{3:D}$ has complexity $O\qty(S_2\prod_{d=1}^D S_d)$. By induction, the total complexity to calculate $p^\theta_{s|t}\qty(x_s^{(1:D)}\middle|x_t^{(1:D)})$ is
\begin{eqnarray*}
  O\qty(\qty(\sum_{d=1}^D S_d)\prod_{d=1}^D S_d).
\end{eqnarray*}

In this paper, we show that by the assumption of Eq. \eqref{EqInterpolarQ}, $p^\theta_{s|t}\qty(x_s^{(1:D)}\middle|x_t^{(1:D)})$ can be calculated by just $O\qty(D\prod_{d=1}^D S_d)$. This will be a big difference in our $D=2$ case. Rewrite Eq. \eqref{EqsOn0tq} to matrix form.
\begin{eqnarray*}
  &&q^{(d)}_{s|0,t}\qty(\cdot\middle|\cdot,x^{(d)}_t)=\qty(1-\mu^{(d)}_{t|s})\qty(I - \mathbf{x}^{(d)}_t\qty(\mathbf{x}^{(d)}_t)^\top) + \qty(\mu^{(d)}_{t|s}\alpha^{(d)}_{t|s}\mathbf{x}^{(d)}_t + \mu^{(d)}_{t|s}\qty(1-\alpha^{(d)}_{t|s})\mathbf{m}_d)\qty(\mathbf{1}-\mathbf{x}^{(d)}_t)^\top\\
  &&+ \qty(\qty(1-\lambda^{(d)}_{t|s})\mathbf{x}^{(d)}_t + \lambda^{(d)}_{t|s}\mathbf{m}_d)\qty(\mathbf{x}^{(d)}_t)^\top\\
  &&=\qty(1-\mu^{(d)}_{t|s}) I - \qty(1-\mu^{(d)}_{t|s})\mathbf{x}^{(d)}_t\qty(\mathbf{x}^{(d)}_t)^\top + \mu^{(d)}_{t|s}\alpha^{(d)}_{t|s}\mathbf{x}^{(d)}_t\mathbf{1}^\top + \mu^{(d)}_{t|s}\qty(1-\alpha^{(d)}_{t|s})\mathbf{m}_d\mathbf{1}^\top - \mu^{(d)}_{t|s}\alpha^{(d)}_{t|s}\mathbf{x}^{(d)}_t\qty(\mathbf{x}^{(d)}_t)^\top\\
  && - \mu^{(d)}_{t|s}\qty(1-\alpha^{(d)}_{t|s})\mathbf{m}_d\qty(\mathbf{x}^{(d)}_t)^\top + \qty(1-\lambda^{(d)}_{t|s})\mathbf{x}^{(d)}_t\qty(\mathbf{x}^{(d)}_t)^\top + \lambda^{(d)}_{t|s}\mathbf{m}_d\qty(\mathbf{x}^{(d)}_t)^\top\\
  &&=\qty(1-\mu^{(d)}_{t|s}) I + \qty(\mu^{(d)}_{t|s} - \lambda^{(d)}_{t|s} - \mu^{(d)}_{t|s}\alpha^{(d)}_{t|s})\qty(\mathbf{x}^{(d)}_t - \mathbf{m}_d)\qty(\mathbf{x}^{(d)}_t)^\top + \mu^{(d)}_{t|s}\alpha^{(d)}_{t|s}\mathbf{x}^{(d)}_t\mathbf{1}^\top + \mu^{(d)}_{t|s}\qty(1-\alpha^{(d)}_{t|s})\mathbf{m}_d\mathbf{1}^\top\\
  &&=\frac{\alpha^{(d)}_{s|0}-\alpha^{(d)}_{t|0}}{1-\alpha^{(d)}_{t|0}} I + \frac{\qty(1-\alpha^{(d)}_{s|0})\qty(1-\alpha^{(d)}_{t|s})\alpha^{(d)}_{t|0}}{\qty(1-\alpha^{(d)}_{t|0})\qty(\alpha^{(d)}_{t|0}+\qty(1-\alpha^{(d)}_{t|0})\left\langle\mathbf{m}_d, \mathbf{x}^{(d)}_t\right\rangle)}\qty(\mathbf{x}^{(d)}_t - \mathbf{m}_d)\qty(\mathbf{x}^{(d)}_t)^\top\\
  && + \qty(\frac{\alpha^{(d)}_{t|s}-\alpha^{(d)}_{t|0}}{1-\alpha^{(d)}_{t|0}}\mathbf{x}^{(d)}_t + \frac{\qty(1-\alpha^{(d)}_{s|0})\qty(1-\alpha^{(d)}_{t|s})}{1-\alpha^{(d)}_{t|0}}\mathbf{m}_d)\mathbf{1}^\top.
\end{eqnarray*}
For all $x_s^{(1:d-1)}$ and $x_0^{(d+1:D)}$, write
\begin{eqnarray*}
  \mathcal{Z}_d\qty(x^{(1:d-1)}_s, x^{(d)}_s, x^{(d+1:D)}_0)=\sum_{x^{(d)}_0}q_{s|0,t}^{(d)}\qty(x_s^{(d)}\middle|x_0^{(d)}, x_t^{(d)})\mathcal{Z}_{d-1}\qty(x_s^{(1:d-1)},x_0^{(d)},x_0^{(d+1:D)})
\end{eqnarray*}
into matrix form
\begin{eqnarray*}
  &&\mathcal{Z}_d\qty(x^{(1:d-1)}_s, \cdot, x^{(d+1:D)}_0) = q_{s|0,t}^{(d)}\qty(\cdot\middle|\cdot, x_t^{(d)})\mathcal{Z}_{d-1}\qty(x_s^{(1:d-1)},\cdot,x_0^{(d+1:D)})\\
  &&=\frac{\alpha^{(d)}_{s|0}-\alpha^{(d)}_{t|0}}{1-\alpha^{(d)}_{t|0}} \mathcal{Z}_{d-1}\qty(x_s^{(1:d-1)},\cdot,x_0^{(d+1:D)})\\
  && + \frac{\qty(1-\alpha^{(d)}_{s|0})\qty(1-\alpha^{(d)}_{t|s})\alpha^{(d)}_{t|0}}{\qty(1-\alpha^{(d)}_{t|0})\qty(\alpha^{(d)}_{t|0}+\qty(1-\alpha^{(d)}_{t|0})\left\langle\mathbf{m}_d, \mathbf{x}^{(d)}_t\right\rangle)}\qty(\mathbf{x}^{(d)}_t - \mathbf{m}_d)\qty(\mathbf{x}^{(d)}_t)^\top\mathcal{Z}_{d-1}\qty(x_s^{(1:d-1)},\cdot,x_0^{(d+1:D)})\\
  && + \qty(\frac{\alpha^{(d)}_{t|s}-\alpha^{(d)}_{t|0}}{1-\alpha^{(d)}_{t|0}}\mathbf{x}^{(d)}_t + \frac{\qty(1-\alpha^{(d)}_{s|0})\qty(1-\alpha^{(d)}_{t|s})}{1-\alpha^{(d)}_{t|0}}\mathbf{m}_d)\mathbf{1}^\top\mathcal{Z}_{d-1}\qty(x_s^{(1:d-1)},\cdot,x_0^{(d+1:D)}),
\end{eqnarray*}
where
\begin{eqnarray*}
  \mathcal{Z}_0\qty(x_0^{(1:D)})\coloneq p^\theta_{0|t}\qty(x_0^{(1:D)}\middle|x_t^{(1:D)}).
\end{eqnarray*}
Note that the complexity is only $O(S_d)$ compared to $O(S_d^2)$ for general case. Since we must do the calculation for all $x_s^{(1:d-1)}$ and $x_0^{(d+1:D)}$, the complexity to get $\mathcal{Z}_d$ is $O\qty(\prod_{d=1}^D S_d)$. Then the total $D$ steps has complexity $O\qty(D\prod_{d=1}^D S_d)$. 

Giving $x^{(d)}_0$, and sampling $t\sim\mathcal{U}(1,T)$ and $x^{(d)}_t\sim \qty(\mathbf{x}^{(d)}_0)^\top Q^{(d)}_{t|0}$, the optimizing target is (see Eq. \eqref{EqDisSpaTarget})
\begin{eqnarray*}
  &&\mathcal{L}^\text{DT}_t(\theta)\coloneq - \mathbb{E}_{q^{(1:D)}_{s|0,t}\qty(x^{(1:D)}_s|x^{(1:D)}_0, x^{(1:D)}_t)}\log p^\theta_{s|t}\qty(x_s^{(1:D)}\middle|x_t^{(1:D)})= - \sum_{x^{(1:D)}_s}\qty[\prod_{d=1}^D q^{(d)}_{s|0,t}\qty(x_s^{(d)}\middle|x_0^{(d)}, x_t^{(d)})]\log p^\theta_{s|t}\qty(x_s^{(1:D)}\middle|x_t^{(1:D)})\\
  &&= - \sum_{x_s^{(D)}}q^{(D)}_{s|0,t}\qty(x_s^{(D)}\middle|x_0^{(D)}, x_t^{(D)})\sum_{x_s^{(D-1)}}q^{(D-1)}_{s|0,t}\qty(x_s^{(D-1)}\middle|x_0^{(D-1)}, x_t^{(D-1)})\cdots\sum_{x_s^{(1)}} q^{(1)}_{s|0,t}\qty(x_s^{(1)}\middle|x_0^{(1)}, x_t^{(1)})\log p^\theta_{s|t}\qty(x_s^{(1:D)}\middle|x_t^{(1:D)}).
\end{eqnarray*}
The complexity is $O\qty(\prod_{d=1}^D S_d)$.

In summary, the total complexity is $O\qty(D\prod_{d=1}^D S_d)$. In our case, $D=2$, so the complexity is $O(S_1S_2)$. Because our network outputs $p^\theta_{0|t}\qty(x^{(1,2)}_0|x^{(1,2)}_t)$, marginalize it to $p^\theta_{0|t}\qty(x^{(1)}_0|x^{(1,2)}_t)$ and $p^\theta_{0|t}\qty(x^{(2)}_0|x^{(1,2)}_t)$ has complexity $O(S_1S_2)$. Therefore, the independent appromiation in Eq. \eqref{Eqp0tCompose} is not necessary in our case.

\section{Reverse sampling}

In https://arxiv.org/pdf/2402.03701, the author derive an explicit expression for $p^\theta_{s|t}\qty(x_s|x_t)$.
\begin{eqnarray*}
  &&\gamma^\theta_{t|s}\coloneqq (\mu_{t|s} - \lambda_{t|s} - \mu_{t|s}\alpha_{t|s})\left\langle \mathbf{x}_t, p^\theta_{0|t}(\cdot|x_t)\right\rangle,\\
  &&p^\theta_{s|t}\qty(\cdot|x_t)=(1-\mu_{t|s})p^\theta_{0|t}(\cdot|x_t) + \qty(\mu_{t|s}\alpha_{t|s} + \gamma^\theta_{t|s})\mathbf{x}_t + \qty(\mu_{t|s}(1-\alpha_{t|s})-\gamma^\theta_{r|s})\mathbf{m}.
\end{eqnarray*}
Then they generalize the result to multiple dimension $D>1$.
\begin{eqnarray*}
  p^{\theta,(d)}_{s|t}\qty(\cdot\middle|x^{(1:D)}_t)=\qty(1-\mu^{(d)}_{t|s})p^{\theta,(d)}_{0|t}\qty(\cdot\middle|x^{(1:D)}_t) + \qty(\mu^{(d)}_{t|s}\alpha^{(d)}_{t|s} + \gamma^{\theta,(d)}_{t|s})\mathbf{x}^{(d)}_t + \qty(\mu^{(d)}_{t|s}\qty(1-\alpha^{(d)}_{t|s})-\gamma^{\theta,(d)}_{r|s})\mathbf{m}_d.
\end{eqnarray*}
However, this generalization depends on the independent approximation Eq. \eqref{Eqp0tCompose}.

$p^{\theta,(d)}_{s|t}$ is used to efficiently sample $x^{(d)}_s$ in the reverse diffusion process. We choose a sampling method not requiring $p^{\theta,(d)}_{s|t}$. That is, sampling $x^{(1:D)}_0$ from $p^\theta_{0|t}\qty(x^{(1:D)}_0\middle|x^{(1:D)}_t)$, and then for each $1\le d\le D$, sampling $x^{(d)}_s$ from $q^{(d)}_{s|0,t}\qty(x^{(d)}_s\middle|x^{(d)}_0,x^{(d)}_t)$. Our method is exact but still efficient. In our case, $D=2$, so sampling $x^{(1:D)}_0$ from $p^\theta_{0|t}$ is of complexity $O(S_1S_2)$, thereby feasible. For large $D$, one may use the independent approximation Eq. \eqref{Eqp0tCompose} to sample $x^{(d)}_0$ from $p^{\theta,(d)}\qty(x^{(d)}_0\middle|x^{(1:D)}_t)$, and then sample $x^{(d)}_s$ from $q^{(d)}_{s|0,t}\qty(x^{(d)}_s\middle|x^{(d)}_0,x^{(d)}_t)$. Note that this requires one additional sampling compared with sampling $x^{(d)}_s$ from $p^{\theta,(d)}_{s|t}\qty(x^{(d)}_s\middle|x^{(1:D)}_t)$. However, this is negligible compared with the network generation of $p^\theta_{0|t}$.

\section{Reparameterize ELBO of continuous-time discrete-space diffusion model}

Assume that all dimensions are independent in the forward process. Then by https://arxiv.org/pdf/2402.03701,
\begin{eqnarray}\label{EqDivParaHatR}
  &&g^{(d)}_t\qty(x^{(d)}\middle|y^{(1:D)})\coloneq\sum_{x^{(d)}_0}\frac{q^{(d)}_{t|0}\qty(x^{(d)}\middle|x^{(d)}_0)}{q^{(d)}_{t|0}\qty(y^{(d)}\middle|x^{(d)}_0)}q^{(d)}_{0|t}\qty(x^{(d)}_0\middle| y^{(1:D)}),\\
  &&\hat{R}_t\qty(x^{(1:D)}\middle|y^{(1:D)})=\sum_{d=1}^D R^{(d)}_t\qty(x^{(d)}\middle| y^{(d)})\delta_{x^{\setminus d},y^{\setminus d}} g^{(d)}_t\qty(x^{(d)}\middle|y^{(1:D)})=\sum_{d=1}^D \frac{R^{(d)}_t\qty(x^{(d)}\middle| y^{(d)})\delta_{x^{\setminus d},y^{\setminus d}}}{g^{(d)}_t\qty(y^{(d)}\middle|x^{(1:D)})}.
\end{eqnarray}
In https://arxiv.org/pdf/2402.03701, the author suggests to parameterize the negative ELBO by
\begin{eqnarray*}
  \mathcal{L}^\text{CT}(\theta)\coloneq T\mathbb{E}_{\mathcal{U}(t;0,T)}\mathbb{E}_{q_t\qty(x^{(1:D)})}\qty[\sum_{y^{(1:D)}\neq x^{(1:D)}}\hat{R}_t^\theta\qty(y^{(1:D)}\middle|x^{(1:D)}) - \sum_{y^{(1:D)}\neq x^{(1:D)}}R_t\qty(y^{(1:D)}\middle|x^{(1:D)})\log\hat{R}_t^\theta\qty(x^{(1:D)}|y^{(1:D)})] + C,
\end{eqnarray*}
where
\begin{eqnarray*}
  &&\hat{R}^\theta_t\qty(x^{(1:D)}\middle|y^{(1:D)})\coloneqq\sum_{d=1}^D R^{(d)}_t\qty(x^{(d)}\middle| y^{(d)})\delta_{x^{\setminus d},y^{\setminus d}} g^{\theta,(d)}_t\qty(x^{(d)}\middle|y^{(1:D)}),\\
  &&g^{\theta,(d)}_t\qty(x^{(d)}\middle|y^{(1:D)})\coloneq\sum_{x^{(d)}_0}\frac{q^{(d)}_{t|0}\qty(x^{(d)}\middle|x^{(d)}_0)}{q^{(d)}_{t|0}\qty(y^{(d)}\middle|x^{(d)}_0)}p^{\theta,(d)}_{0|t}\qty(x^{(d)}_0\middle| y^{(1:D)}).
\end{eqnarray*}
However, this parameterization needs an expensive network inference to get $p^{\theta,(d)}_{0|t}\qty(\cdot\middle|y^{(1:D)})$ for each $y^{(1:D)}$. Both https://arxiv.org/pdf/2205.14987 and https://arxiv.org/pdf/2402.03701 avoid the multiple passes by changing the expectation variable
through importance sampling. We propose a simpler method as follows.

By Eq. \eqref{EqDivParaHatR}, define
\begin{eqnarray}
  \tilde{R}^\theta_t\qty(x^{(1:D)}\middle|y^{(1:D)})\coloneq\sum_{d=1}^D \frac{R^{(d)}_t\qty(x^{(d)}\middle| y^{(d)})\delta_{x^{\setminus d},y^{\setminus d}}}{g^{\theta,(d)}_t\qty(y^{(d)}\middle|x^{(1:D)})}.
\end{eqnarray}
Replace $\hat{R}_t\qty(x^{(1:D)}\middle|y^{(1:D)})$ by $\tilde{R}_t\qty(x^{(1:D)}\middle|y^{(1:D)})$ in $\mathcal{L}^\text{CT}(\theta)$.
\begin{eqnarray}\label{EqBetterNgELBO}
  \tilde{\mathcal{L}}^\text{CT}(\theta)\coloneq T\mathbb{E}_{\mathcal{U}(t;0,T)}\mathbb{E}_{q_t\qty(x^{(1:D)})}\qty[\sum_{y^{(1:D)}\neq x^{(1:D)}}\hat{R}_t^\theta\qty(y^{(1:D)}\middle|x^{(1:D)}) - \sum_{y^{(1:D)}\neq x^{(1:D)}}R_t\qty(y^{(1:D)}\middle|x^{(1:D)})\log\tilde{R}_t^\theta\qty(x^{(1:D)}|y^{(1:D)})] + C.
\end{eqnarray}
Then all one needs is to calculate $g^\theta_t(y|x)$, which only needs a single pass for $p^\theta_{0|t}(\cdot|x)$. Even though https://arxiv.org/pdf/2402.03701 derives Eq. \eqref{EqDivParaHatR}, they do not use it to get Eq. \eqref{EqBetterNgELBO}.

By https://arxiv.org/pdf/2402.03701,
\begin{eqnarray*}
  \sum_{y^{(1:D)}\neq x^{(1:D)}}\hat{R}_t^\theta\qty(y^{(1:D)}\middle|x^{(1:D)})=\sum_{d=1}^D\left\langle\mathbf{x}^{(d)},\mathbf{m}_d\right\rangle\left\langle\mathbf{1},g^{\theta,(d)}_t\qty(\cdot\middle|x^{(1:D)})\right\rangle + C.
\end{eqnarray*}
We now simplify the second term.
\begin{eqnarray*}
  &&- \sum_{y^{(1:D)}\neq x^{(1:D)}}R_t\qty(y^{(1:D)}\middle|x^{(1:D)})\log\tilde{R}_t^\theta\qty(x^{(1:D)}\middle|y^{(1:D)})\\
  &&=-\sum_{d=1}^D\sum_{y^{(d)}\neq x^{(d)}}R^{(d)}_t\qty(y^{(d)}\middle|x^{(d)})\log\frac{R_t^{(d)}\qty(x^{(d)}|y^{(d)})}{g^{\theta,(d)}_t\qty(y^{(d)}\middle|x^{(1:D)})}\\
  &&=\sum_{d=1}^D\sum_{y^{(d)}\neq x^{(d)}}R^{(d)}_t\qty(y^{(d)}\middle|x^{(d)})\log g^{\theta,(d)}_t\qty(y^{(d)}\middle|x^{(1:D)}) + C\\
  &&=\sum_{d=1}^D \left\langle R^{(d)}_t\qty(\cdot\middle|x^{(d)}),\log g^{\theta,(d)}_t\qty(\cdot\middle|x^{(1:D)})\right\rangle + C\\
  &&=\sum_{d=1}^D \left\langle \mathbf{m}_d^\top,\log g^{\theta,(d)}_t\qty(\cdot\middle|x^{(1:D)})\right\rangle + C.
\end{eqnarray*}
Thus,
\begin{eqnarray*}
  \tilde{\mathcal{L}}^\text{CT}(\theta)=T\mathbb{E}_{\mathcal{U}(t;0,T)}\mathbb{E}_{q_t\qty(x^{(1:D)})}\sum_{d=1}^D\qty[\left\langle\mathbf{x}^{(d)},\mathbf{m}_d\right\rangle\left\langle\mathbf{1},g^{\theta,(d)}_t\qty(\cdot\middle|x^{(1:D)})\right\rangle + \left\langle \mathbf{m}_d^\top,\log g^{\theta,(d)}_t\qty(\cdot\middle|x^{(1:D)})\right\rangle] + C.
\end{eqnarray*}

\section{MCMC correction}
In both https://arxiv.org/pdf/2205.14987 and https://arxiv.org/pdf/2402.03701, the author suggest to use MCMC correction. In their case, $D$ is large (high dimensional case). Thus, it is expensive to express the $D$-dimensional $q_0\qty(\cdot)$. They choose to sample $x^{(1:D)}_0$ and $x^{(1:D)}_t$ from $q_{0,t}\qty(x^{(1:D)}_0, x^{(1:D)}_t)$, and use
\begin{eqnarray*}
  \mathcal{L}^\text{CE}_t(\theta)\gets -\log p^\theta_{0|t}\qty(x^{(1:D)}_0\middle|x^{(1:D)}_t).
\end{eqnarray*}
as a correction to the loss function.

Note that
\begin{eqnarray*}
  &&q_{t|0}\qty(x^{(1:D)}_t\middle|x^{(1:D)}_0) = \prod_{d=1}^D q^{(d)}_{t|0}\qty(x^{(d)}_t\middle|x^{(d)}_0),\\
  &&q^{(d)}_{t|0}\qty(x^{(d)}_t\middle|\cdot) = \alpha^{(d)}_t \mathbf{x}^{(d)}_t + \qty(1 - \alpha^{(d)}_t)\left\langle\mathbf{x}^{(d)}_t, \mathbf{m}_d\right\rangle\mathbf{1}.
\end{eqnarray*}
In our case, $D = 2$. Then
\begin{eqnarray*}
  q_{0, t}\qty(\cdot, x^{(1:2)}_t) = q^{(1)}_{t|0}\qty(x^{(1)}_t\middle|\cdot) \circ q_0\qty(\cdot) \circ \qty(q^{(2)}_{t|0}\qty(x^{(2)}_t\middle|\cdot))^\top.
\end{eqnarray*}
Normalize $q_{0, t}\qty(\cdot, x^{(1:2)}_t)$ to $q_{0|t}\qty(\cdot\middle|x^{(1:2)}_t)$. Then a better MCMC correction is
\begin{eqnarray*}
  \tilde{\mathcal{L}}^\text{CE}_t(\theta)\gets -\mathbb{E}_{q_{0|t}\qty(x^{(1:D)}_0\middle|x^{(1:D)}_t)}\log p^\theta_{0|t}\qty(x^{(1:D)}_0\middle|x^{(1:D)}_t).
\end{eqnarray*}

\section{Sampling}
CRISPR data have lots of duplicate for some reads. Suppose there are $N$ read types with duplicates $\qty{c_i}_{i=1}^N$. Generally, the wildtype $c_1$ dominates the total population. Thus, the rare case cannot be trained efficiently. One goodness of diffusion model is that it diffuse the wildtype mass to non-wildtype case, which mitigates the unbalance of training data.

\section{Validation}
To validate the accuracy of diffuser model, one must sample lots of data from the diffuser, which is quite inefficient, especially when there are lots of time steps. Since each step of the reverse process generates an estimation of $p^\theta_0\qty(\cdot\middle|x^{(1:D)}_t)$, we use the approximation
\begin{eqnarray*}
  p^\theta_0\qty(x^{(1:D)}_0) \approx \frac{\sum_{t=1}^T p^\theta_0\qty(\cdot\middle|x^{(1:D)}_t)}{T}.
\end{eqnarray*}
The valid loss is defined as the negative cross entropy $\mathbb{E}_{q_0\qty(x^{(1:D)}_0)}\qty[\log p_0^\theta\qty(x^{(1:D)}_0)]$.

\section{Noise scheduler}

https://arxiv.org/pdf/2402.03701 collects three noise scheduler in previous works. However, these noise scheduler is given with non-time-homogeneous diffusion. In this section, we derive their time-homogeneous counterpart. We use
\begin{eqnarray*}
  \alpha_{t|0}= e^{-\int_0^t \beta_s ds} = e^{-u}.
\end{eqnarray*}

\subsection{The linear noise scheduler}
The linear noise scheduler is in https://arxiv.org/pdf/2006.11239.
\begin{eqnarray*}
  &&e^{-u} = 1 - \frac{t}{N},\\
  &&u = \log\frac{T}{T-t}.
\end{eqnarray*}
Because $\lim_{t\to T^-}u = +\infty$, we set $u=+\infty$ for $t=T$.

\subsection{The cosine noise scheduler}
The cosine noise scheduler is in https://arxiv.org/pdf/2102.05379.
\begin{eqnarray*}
  &&e^{-u} = \frac{\cos\qty(\frac{t/T+a}{1+a}\frac{\pi}{2})}{\cos\qty(\frac{a}{1+a}\frac{\pi}{2})},\\
  &&u = \log \frac{\cos\qty(\frac{a}{1+a}\frac{\pi}{2})}{\cos\qty(\frac{t/T+a}{1+a}\frac{\pi}{2})},
\end{eqnarray*}
where $a=0.008$ by defaults. Because $\lim_{t\to T^-}u = +\infty$, we set $u=+\infty$ for $t=T$.

\subsection{The exponential noise scheduler}
The exponential noise scheduler is in https://arxiv.org/pdf/2205.14987.
\begin{eqnarray*}
  &&e^{-u} = e^{a\qty(1-b^{t/T})},\\
  &&u = a\qty(b^{t/T} - 1),
\end{eqnarray*}
where $a=b=5$ by defaults.

\section{Algorithms}

\begin{algorithm}
  \caption{Training: {\color{red} red: discrete-time step}; {\color{blue} blue: continuous-time step}.}
  \label{TrainAlg}
  \begin{algorithmic}
    \Require $\mathbf{m}$, $q_0\qty(x^{(1:D)})$, $\lambda$, $T$, $0<t_1<t_2<t_3<\cdots<t_n=T$.
    \Repeat
    \State Draw $x^{(1:D)}_0\sim q_0\qty(x^{(1:D)})$.
    \State Draw {\color{red} $t\sim\mathcal{U}\qty(\qty{t_i}_{i=1}^n)$} or {\color{blue} $t\sim\mathcal{U}(t;0,T)$}.
    \State $\alpha_{t|s}\gets e^{-(t-s)}$.
    \For{$d$ in $[1,D]$}
    \State Draw $y\sim \mathbf{m}_d$.
    \State Draw $b\sim \text{Bernoulli}\qty(\alpha_{t|0})$.
    \State $x^{(d)}_t \gets bx^{(d)}_0 + (1-b)y$.
    \EndFor
    \State Use network to predict $p^\theta_{0|t}\qty(\cdot\middle|x^{(1:D)}_t)$.
    \For{$d$ in $[1,D]$}
    \State \State $a^{(d)}_3\gets 1+\qty(\qty(\alpha^{(d)}_{t|0})^{-1}-1)\left\langle\mathbf{m}_d, \mathbf{x}^{(d)}_t\right\rangle$.
    \EndFor
    \State {\color{red} $s\gets t-1$.}
    \State {\color{red} $\mathcal{Z}_0\qty(x_0^{(1:D)})\gets p^\theta_{0|t}\qty(x_0^{(1:D)}\middle|x_t^{(1:D)})$.}
    \For{$d$ in $[1,D]$}
    \State {\color{red} $a_0\gets \alpha^{(d)}_{s|0}\qty(1-\alpha^{(d)}_{t|s})$.}
    \State {\color{red} $a_1\gets \qty(1-\alpha^{(d)}_{s|0})\qty(1-\alpha^{(d)}_{t|s})$.}
    \State {\color{red} $a_2\gets \qty(1-\alpha^{(d)}_{s|0})\alpha^{(d)}_{t|s}$.}
    \State {\color{red} $b_1\qty(x_s^{(1:d-1)},x^{(d)}_t,x_0^{(d+1:D)})\gets \frac{a_1}{a^{(d)}_3}\mathcal{Z}_{d-1}\qty(x_s^{(1:d-1)},x^{(d)}_t,x_0^{(d+1:D)})$.}
    \State {\color{red} $b_2\qty(x_s^{(1:d-1)},x_0^{(d+1:D)})\gets \mathbf{1}^\top\mathcal{Z}_{d-1}\qty(x_s^{(1:d-1)},\cdot,x_0^{(d+1:D)})$.}
    \State {\color{red} $\mathcal{Z}_d\qty(x_s^{(1:d-1)},\cdot,x_0^{(d+1:D)})\gets \frac{a_0 \mathcal{Z}_{d-1}\qty(x_s^{(1:d-1)},\cdot,x_0^{(d+1:D)}) + b_1\qty(x_s^{(1:d-1)},x^{(d)}_t,x_0^{(d+1:D)}) \qty(\mathbf{x}^{(d)}_t - \mathbf{m}_d) + b_2\qty(x_s^{(1:d-1)},x_0^{(d+1:D)})\qty(a_2\mathbf{x}^{(d)}_t + a_1\mathbf{m}_d)}{1-\alpha^{(d)}_{t|0}}$.}
    \EndFor
    \State {\color{red} $p^\theta_{s|t}\qty(x^{(1:D)}_s\middle|x^{(1:D)}_t)=\mathcal{Z}_D\qty(x^{(1:D)}_s)$.}
    \State {\color{red} $q^{(d)}_{s|0,t}\qty(\cdot\middle|x^{(d)}_0,x^{(d)}_t)\gets \frac{\qty(\alpha_{t|s}\mathbf{x}^{(b)}_t + \qty(1-\alpha_{t|s})\left\langle\mathbf{m}_b,\mathbf{x}^{(b)}_t\right\rangle\mathbf{1})\circ\qty(\alpha_{s|0}\mathbf{x}^{(b)}_0+\qty(1-\alpha_{s|0})\mathbf{m}_b)}{\alpha_{t|0}\left\langle\mathbf{x}^{(b)}_t, \mathbf{x}^{(b)}_0\right\rangle + (1-\alpha_{t|0})\left\langle\mathbf{m}_b, \mathbf{x}^{(b)}_t\right\rangle}$.}
    \State
    {\color{red}
    \begin{eqnarray*}
      \mathcal{L}^\text{DT}_t(\theta)\gets - \sum_{x_s^{(D)}}q^{(D)}_{s|0,t}\qty(x_s^{(D)}\middle|x_0^{(D)}, x_t^{(D)})\sum_{x_s^{(D-1)}}q^{(D-1)}_{s|0,t}\qty(x_s^{(D-1)}\middle|x_0^{(D-1)}, x_t^{(D-1)})\cdots\sum_{x_s^{(1)}} q^{(1)}_{s|0,t}\qty(x_s^{(1)}\middle|x_0^{(1)}, x_t^{(1)})\log p^\theta_{s|t}\qty(x_s^{(1:D)}\middle|x_t^{(1:D)}).
    \end{eqnarray*}
    }
    \State
    {\color{blue}
    \begin{eqnarray*}
      g^{\theta,(d)}_t\qty(\cdot\middle|x^{(1:D)}_t)\gets \frac{1}{\left\langle \mathbf{x}^{(d)}_t,\mathbf{m}_d\right\rangle}\qty[\qty(1-\frac{\left\langle p^{\theta,(d)}_{0|t}\qty(\cdot\middle|x^{(1:D)}_t),x^{(d)}_t\right\rangle}{a^{(d)}_3})\mathbf{m}_d + \frac{\alpha_{t|0}}{1-\alpha_{t|0}}p^{\theta,(d)}_{0|t}\qty(\cdot\middle|x^{(1:D)}_t)]\circ \qty(\mathbf{1}-\mathbf{x}^{(d)}_t) + \mathbf{x}^{(d)}_t.
    \end{eqnarray*}
    }
    \State {\color{blue} $\tilde{\mathcal{L}}^\text{CT}_t(\theta)\gets \sum_{d=1}^D\qty[\left\langle\mathbf{x}^{(d)}_t,\mathbf{m}_d\right\rangle\left\langle\mathbf{1},g^{\theta,(d)}_t\qty(\cdot\middle|x^{(1:D)}_t)\right\rangle + \left\langle \mathbf{m}_d^\top,\log g^{\theta,(d)}_t\qty(\cdot\middle|x^{(1:D)}_t)\right\rangle]$.}
    \State
    \begin{eqnarray*}
      &&q^{(d)}_{t|0}\qty(x^{(d)}_t\middle|\cdot) \gets \alpha^{(d)}_t \mathbf{x}^{(d)}_t + \qty(1 - \alpha^{(d)}_t)\left\langle\mathbf{x}^{(d)}_t, \mathbf{m}_d\right\rangle\mathbf{1},\\
      &&q_{0,t}\qty(x^{(1:D)}_0, x^{(1:D)}_t) \gets q_0\qty(x^{(1:D)}_0)\prod_{d=1}^D q^{(d)}_{t|0}\qty(x^{(d)}_t\middle|x^{(d)}_0),\\
      &&q_t\qty(x^{(1:D)}_t) \gets \sum_{x^{(1:D)}_0}q_{0,t}\qty(x^{(1:D)}_0, x^{(1:D)}_t),\\
      &&q_{0|t}\qty(x^{(1:D)}_0\middle|x^{(1:D)}_t) \gets q_{0,t}\qty(x^{(1:D)}_0, x^{(1:D)}_t) / q_t\qty(x^{(1:D)}_t).
    \end{eqnarray*}
    \State $\tilde{\mathcal{L}}^\text{CE}_t(\theta)\gets -\mathbb{E}_{q_{0|t}\qty(x^{(1:D)}_0\middle|x^{(1:D)}_t)}\log p^\theta_{0|t}\qty(x^{(1:D)}_0\middle|x^{(1:D)}_t)$.
    \State {\color{red} $\nabla_\theta\qty[\mathcal{L}^\text{DT}_t(\theta) + \lambda \mathcal{L}^\text{CE}_t(\theta)]$} or {\color{blue} $\nabla_\theta\qty[\mathcal{L}^\text{CT}_t(\theta) + \lambda \mathcal{L}^\text{CE}_t(\theta)]$}.
    \Until{convergence}
  \end{algorithmic}
\end{algorithm}

\begin{algorithm}
  \caption{Sampling.}
  \label{SampleAlg}
  \begin{algorithmic}
    \Require $\mathbf{m}$. $0<t_1<t_2<t_3<\cdots<t_n=T$.
    \Ensure use\_MCMC. Step size $\Delta t$. Total step $N$.
    \State Draw $x^{(1:D)}_{t_n}\sim \mathbf{m}$.
    \For{$i\in\set{n,\cdots,1}$}
    \State Get $p^\theta_{0|t_i}\qty(\cdot\middle|x^{(1:D)}_{t_i})$ by network.
    \State Draw $x^{(1:D)}_0\sim p^\theta_{0|t}\qty(\cdot\middle|x^{(1:D)}_{t_i})$.
    \State $q^{(d)}_{t_{i-1}|0,t_i}\qty(\cdot\middle|x^{(d)}_0,x^{(d)}_{t_i})\gets \frac{\qty(\alpha_{t_i|t_{i-1}}\mathbf{x}^{(b)}_{t_i} + \qty(1-\alpha_{t_i|t_{i-1}})\left\langle\mathbf{m}_b,\mathbf{x}^{(b)}_{t_i}\right\rangle\mathbf{1})\circ\qty(\alpha_{t_{i-1}|0}\mathbf{x}^{(b)}_0+\qty(1-\alpha_{t_{i-1}|0})\mathbf{m}_b)}{\alpha_{t_i|0}\left\langle\mathbf{x}^{(b)}_{t_i}, \mathbf{x}^{(b)}_0\right\rangle + (1-\alpha_{t_i|0})\left\langle\mathbf{m}_b, \mathbf{x}^{(b)}_{t_i}\right\rangle}$.
    \State Draw $x^{(1:D)}_{t_{i-1}}\sim q^{(1:D)}_{t_{i-1}|0,t_i}\qty(\cdot\middle|x^{(1:D)}_0,x^{(1:D)}_{t_i})=\prod_{d=1}^D q^{(d)}_{t_{i-1}|0,t_i}\qty(\cdot\middle|x^{(d)}_0,x^{(d)}_{t_i})$.
    \If{use\_MCMC}:
    \For{$m$ from $1$ to $N$}
    \State Get $p^\theta_{0|t_{i-1}}\qty(\cdot\middle|x^{(1:D)}_{t_{i-1}})$ by network.
    \State $\bar{p}^{\theta,(d)}_{\Delta t}\qty(\cdot|x^{(1:D)}_{t_{i-1}})\gets \Delta t\qty[\qty(2-\frac{\left\langle p^{\theta,(d)}_{0|t_{i-1}}\qty(\cdot\middle|x^{(1:D)}_{t_{i-1}}),\mathbf{x}^{(d)}_{t_{i-1}}\right\rangle}{1+\qty(\qty(\alpha_{t|0})^{-1}-1)\left\langle \mathbf{x}^{(d)}_{t_{i-1}},\mathbf{m}_d\right\rangle})\mathbf{m}_d + \frac{\alpha_{t|0}}{1-\alpha_{t|0}}p^{\theta,(d)}_{0|t_{i-1}}\qty(\cdot\middle|x^{(1:D)}_{t_{i-1}})]\circ\qty(\mathbf{1}-\mathbf{x}^{(d)}_{t_{i-1}})$.
    \State Draw $x^{(d)}_{t_{i-1}}\sim \bar{p}^{\theta,(d)}_{\Delta t}\qty(\cdot|x^{(1:D)}_{t_i}) + \qty(1-\sum_{y}\bar{p}^{\theta,(d)}_{\Delta t}\qty(y|x^{(1:D)}_{t_{i-1}}))\mathbf{x}^{(d)}_{t_{i-1}}$.
    \EndFor
    \EndIf
    \EndFor
  \end{algorithmic}
\end{algorithm}

\appendix

\section{Maximize ELBO is equivalent to maximize enhanced data}
By https://arxiv.org/pdf/2402.04384,
\begin{eqnarray*}
  &&\mathbb{E}_{q_0(x_0)}\qty[\log p_0^\theta(x_0)]=\mathbb{E}_{q_{0:T}(x_{0:T})}\qty[\log p_0^{\theta}(x_0)]=\mathbb{E}_{q_0(x_0)}\mathbb{E}_{q_{1:T|0}(x_{1:T}|x_0)}\qty[\log p_0^{\theta}(x_0)]\\
  &&=\mathbb{E}_{q_0(x_0)}\mathbb{E}_{q_{1:T|0}(x_{1:T}|x_0)}\qty[\log \frac{p_{0:T}^{\theta}(x_{0:T})q_{1:T|0}(x_{1:T}|x_0)}{p_{1:T|0}^{\theta}(x_{1:T}|x_0)q_{1:T|0}(x_{1:T}|x_0)}]\\
  &&=\mathbb{E}_{q_0(x_0)}\underbrace{\mathbb{E}_{q_{1:T|0}(x_{1:T}|x_0)}\qty[\log \frac{p_{0:T}^{\theta}(x_{0:T})}{q_{1:T|0}(x_{1:T}|x_0)}]}_\text{ELBO} + \mathbb{E}_{q_0(x_0)}\underbrace{\mathbb{E}_{q_{1:T|0}(x_{1:T}|x_0)}\qty[\log \frac{q_{1:T|0}(x_{1:T}|x_0)}{p_{1:T|0}^{\theta}(x_{1:T}|x_0)}]}_{\text{KL}\qty(q_{1:T|0}(x_{1:T}|x_0)||p_{1:T|0}^{\theta}(x_{1:T}|x_0))\ge 0}\\
  &&\ge \mathbb{E}_{q_0(x_0)}\underbrace{\mathbb{E}_{q_{1:T|0}(x_{1:T}|x_0)}\qty[\log \frac{p_{0:T}^{\theta}(x_{0:T})}{q_{1:T|0}(x_{1:T}|x_0)}]}_\text{ELBO}\\
  &&=\underbrace{\mathbb{E}_{q_{0:T}(x_{0:T})}\qty[\log p_{0:T}^{\theta}(x_{0:T})]}_\text{unweighted DDPM loss function} - \underbrace{\mathbb{E}_{q_{0:T}(x_{0:T})}\qty[\log q_{1:T|0}(x_{1:T}|x_0)]}_\text{constant independent of $\theta$}\\
  &&=\mathbb{E}_{q_{0:T}(x_{0:T})}\qty[\log \qty(p_T(x_T)\prod_{t=1}^{T} p_{t-1|t}^{\theta}(x_{t-1}|x_t))] + C\\
  &&=\sum_{t=1}^T\mathbb{E}_{q_{t-1,t}(x_{t-1},x_t)}\log p_{t-1|t}^\theta(x_{t-1}|x_t) + \underbrace{\mathbb{E}_{q_T(x_T)} \log p_T(x_T)}_\text{constant independent of $\theta$} + C\\
  &&=T\sum_{t=1}^T \frac{1}{T}\mathbb{E}_{q_{t-1,t}(x_{t-1},x_t)}\log p_{t-1|t}^\theta(x_{t-1}|x_t) + C\\
  &&=T\mathbb{E}_{\mathcal{U}(t;1,T)}\mathbb{E}_{q_{t-1,t}(x_{t-1},x_t)}\log p_{t-1|t}^\theta(x_{t-1}|x_t) + C
\end{eqnarray*}

\section{Evaluation of ELBO}

The ground truth of $q_0(x_0)$ is unknow, but can be approximated by selecting $x_0$ from data.
\begin{eqnarray*}
  \mathbb{E}_{q_{t-1,t}(x_{t-1},x_t)}\log p_{t-1|t}^\theta(x_{t-1}|x_t) = \mathbb{E}_{q_{0,t-1,t}(x_0,x_{t-1},x_t)}\log p_{t-1|t}^\theta(x_{t-1}|x_t)
\end{eqnarray*}
implies a naive sampling of $x_{t-1}$ and $x_t$ given $x_0$, which is easy but unstable (larger variance).

What about sampling nothing?
\begin{eqnarray*}
  \mathbb{E}_{q_{0,t-1,t}(x_0,x_{t-1},x_t)}\log p_{t-1|t}^\theta(x_{t-1}|x_t) = \mathbb{E}_{q_0(x_0)}\iint q_{t-1,t|0}(x_{t-1},x_t|x_0)\log p_{t-1|t}^\theta(x_{t-1}|x_t) dx_{t-1}dx_t.
\end{eqnarray*}
It is feasible as long as the double integration (or summation for discrete space) can be efficiently calculated.

The middle way is to only sample $x_t$.
\begin{eqnarray*}
  &&\mathbb{E}_{q_{0,t-1,t}(x_0,x_{t-1},x_t)}\log p_{t-1|t}^\theta(x_{t-1}|x_t) = \mathbb{E}_{q_{0,t}(x_0,x_t)}\int q_{t-1|0,t}(x_{t-1}|x_0,x_t)\log p_{t-1|t}^\theta(x_{t-1}|x_t) dx_{t-1}\\
  &&= \mathbb{E}_{q_{0,t}(x_0,x_t)}\int q_{t-1|0,t}(x_{t-1}|x_0,x_t)\log \qty(\int q_{t-1|0,t}(x_{t-1}|x'_0,x_t) p_{0|t}^\theta(x'_0|x_t) dx'_0) dx_{t-1}.
\end{eqnarray*}
For continuous space, the network predicts $x'_0$ but not its distribution. Thus,
\begin{eqnarray*}
  &&\mathbb{E}_{q_{0,t}(x_0,x_t)}\int q_{t-1|0,t}(x_{t-1}|x_0,x_t)\log \qty(\int q_{t-1|0,t}(x_{t-1}|x'_0,x_t) p_{0|t}^\theta(x'_0|x_t) dx'_0) dx_{t-1}\\
  &&=\mathbb{E}_{q_{0,t}(x_0,x_t)}\int q_{t-1|0,t}(x_{t-1}|x_0,x_t)\log \qty(\int q_{t-1|0,t}(x_{t-1}|x'_0,x_t) \delta(x_0^\theta) dx'_0) dx_{t-1}\\
  &&=\mathbb{E}_{q_{0,t}(x_0,x_t)}\int q_{t-1|0,t}(x_{t-1}|x_0,x_t)\log q_{t-1|0,t}(x_{t-1}|x^\theta_0,x_t) dx_{t-1}.
\end{eqnarray*}
For discrete space,
\begin{eqnarray}\label{EqDisSpaTarget}
  &&\mathbb{E}_{q_{0,t-1,t}(x_0,x_{t-1},x_t)}\log p_{t-1|t}^\theta(x_{t-1}|x_t) = \mathbb{E}_{q_{0,t}(x_0,x_t)}\sum_{x_{t-1}} q_{t-1|0,t}(x_{t-1}|x_0,x_t)\log p_{t-1|t}^\theta(x_{t-1}|x_t)\nonumber\\
  &&= \mathbb{E}_{q_{0,t}(x_0,x_t)}\sum_{x_{t-1}} q_{t-1|0,t}(x_{t-1}|x_0,x_t)\log \qty(\sum_{x'_0} q_{t-1|0,t}(x_{t-1}|x'_0,x_t) p_{0|t}^\theta(x'_0|x_t)).
\end{eqnarray}

\section{Parameterization of $p^\theta_{t-1|t}(x_{t-1}|x_t)$}

Continuous-space discrete-time diffusion models has
\begin{eqnarray*}
  &&q_{t|t-1}(x_t|x_{t-1})\coloneq\mathcal{N}\qty(x_t;\sqrt{\alpha_t}x_{t-1}, 1-\alpha_t),\\
  &&\alpha_{t|s}\coloneq\prod_{i=s+1}^t \alpha_i,\\
  &&q_{t|0}(x_t|x_0)=\mathcal{N}\qty(x_t;\sqrt{\alpha_{t|0}}x_0, 1-\alpha_{t|0}),\\
  &&q_{t-1|0,t}(x_{t-1}|x_0,x_t)=\mathcal{N}\qty(x_{t-1};\frac{\sqrt{\alpha_{t-1|0}}(1-\alpha_t)}{1-\alpha_{t|0}}x_0 + \frac{(1-\alpha_{t-1|0})\sqrt{\alpha_t}}{1-\alpha_{t|0}}x_t, \frac{(1-\alpha_{t-1|0})(1-\alpha_t)}{1-\alpha_{t|0}}).
\end{eqnarray*}
$p_{t-1|t}^\theta(x_{t-1}|x_t)$ is parameterized by
\begin{eqnarray*}
  p_{t-1|t}^\theta(x_{t-1}|x_t) \coloneq q\qty(x_{t-1}|x_t, x^\theta_0).
\end{eqnarray*}
There are two ways to parameterize $x^\theta_0$ (https://arxiv.org/pdf/2402.04384). 

For $0$-parameterization, $x_0^\theta=x_0^\theta(x_t,t)$. That means the network predict $x_0^\theta$ from $x_t$ and $t$ directly.

For $\epsilon$-parameterisation, the network predicts $\epsilon^\theta(x_t,t)$ such that
\begin{eqnarray*}
  x_t\approx \sqrt{\alpha_{t|0}} x_0 + \qty(1-\alpha_{t|0}) \epsilon^\theta(x_t,t).
\end{eqnarray*}
Then $x_0^\theta$ can be parameterized as
\begin{eqnarray*}
  x^\theta_0 = \frac{x_t - \qty(1-\alpha_{t|0}) \epsilon^\theta(x_t,t)}{\sqrt{\alpha_{t|0}}}.
\end{eqnarray*}

For discrete-space discrete-time diffusion models, thera are three parameterizations of $p^\theta_{t-1|t}(x_{t-1}|x_t)$ (https://arxiv.org/pdf/2402.03701).
\begin{enumerate}
  \item The network predicts $p^\theta_{t-1|t}(\cdot|x_t)$ directly, which does not reuse any known distribution from the forward process, and hence is less effective in practice.
  \item The network predicts $p^\theta_{0|t}(\cdot|x_t)$.
  \begin{enumerate}
    \item Parameterize $p^\theta_{t-1|t}(x_{t-1}|x_t)$ as
    \begin{eqnarray*}
      p^\theta_{t-1|t}(x_{t-1}|x_t)\coloneq q_{t-1|0,t}\qty(\mathbf{x}_{t-1}\middle|p^\theta_{0|t}(\cdot|x_t),\mathbf{x}_t),
    \end{eqnarray*}
    where $\mathbf{x}_t$ is the one-hot vector of $x_t$. Some heuristics have been proposed in https://arxiv.org/pdf/2402.03701. However, those can have a large gap to the true $q_{t-1|t}(x_{t−1}|x_t)$, leading to an inaccurate sampling process.
    \item Parameterize $p^\theta_{t-1|t}(x_{t-1}|x_t)$ as
    \begin{eqnarray*}
      p^\theta_{t-1|t}(x_{t-1}|x_t)\coloneq \sum_{x_0}q_{t-1|0,t}\qty(x_{t-1}|x_0,x_t)p^\theta_{0|t}(x_0|x_t).
    \end{eqnarray*}
  \end{enumerate}
\end{enumerate}

\section{Continuous-time discrete diffusion}
In https://arxiv.org/pdf/2205.14987, the derivation of continuous-time ELBO treat $\mathbb{E}_{q_{k+1|k}(x_{k+1}|x_k)}\qty[\log p_{k|k+1}^\theta(x_k|x_{k+1})]$ for $k>0$ and $k=0$ differently. This is not necessary. Actually, the negative ELBO (or VLB)
\begin{eqnarray*}
  \mathcal{L}^\text{DT}(\theta) = -\Delta t\sum_{k=0}^{K-1}\mathbb{E}_{q_k(x_k)}\qty[\hat{R}_k^\theta(x_k|x_k) + \sum_{x_{k+1}\neq x_k}R_k(x_{k+1}|x_k)\log\hat{R}_k^\theta(x_k|x_{k+1})] + o(\Delta t) + C.
\end{eqnarray*}
Then
\begin{eqnarray*}
  &&\mathcal{L}^\text{CT}(\theta) = \lim_{\Delta t\to 0}\mathcal{L}^\text{DT}(\theta) = - \int_0^T \mathbb{E}_{q_t(x)}\qty[\hat{R}_t^\theta(x|x) + \sum_{y\neq x}R_t(y|x)\log\hat{R}_t^\theta(x|y)]dt + C\\
  &&=\int_0^T \mathbb{E}_{q_t(x)}\qty[\sum_{y\neq x}\hat{R}_t^\theta(y|x) - \sum_{y\neq x}R_t(y|x)\log\hat{R}_t^\theta(x|y)]dt + C\\
  &&=T\mathbb{E}_{\mathcal{U}(t;0,T)}\mathbb{E}_{q_t(x)}\qty[\sum_{y\neq x}\hat{R}_t^\theta(y|x) - \sum_{y\neq x}R_t(y|x)\log\hat{R}_t^\theta(x|y)] + C.
\end{eqnarray*}

\section{Relate the time for $R_t=\beta_t R_b$}

The continuous-time discrete markov process $\frac{d q_t}{dt}=q_t R_t$ has a closed form solution $q_t = q_0 e^{\int_0^t R_s ds}$ if for all $s$ and $t$, $R_s$ and $R_t$ commute. In https://arxiv.org/pdf/2205.14987, the authors suggest to use $R_t\coloneqq \beta_t R_b$, where $R_b$ is a base transition rate, and $\beta_t$ is a time-dependent scalar. They also claim that $R_t$ can be transformed to time-homogeneous process by rescale time but do not give more details. Here is more details.

Define the time scale $u=F(t)\coloneqq\int_0^t\beta_s ds$. Assume that $\beta_s > 0$ for all $s$. Then $F^{-1}(u)$ is well-defined. By variable substitution $s=F^{-1}(v)$, the rescaled process $q_{F^{-1}(u)}$ satisfies
\begin{eqnarray*}
  &&q_{F^{-1}(u)}\coloneq q_0 e^{R_b \int_0^{F^{-1}(u)}\beta_s ds} = q_0 e^{R_b \int_0^{F\circ F^{-1}(u)}\beta_{F^{-1}(v)} dF^{-1}(v)}=q_0 e^{R_b \int_0^{F\circ F^{-1}(u)}\beta_{F^{-1}(v)} dF^{-1}(v)}=q_0 e^{R_b \int_0^u\beta_{F^{-1}(v)} \partial_v F^{-1}(v)dv}\\
  &&q_0 e^{R_b \int_0^u\beta_{F^{-1}(v)} \partial^{-1}_{F^{-1}(v)} F\qty(F^{-1}(v))dv}=q_0 e^{R_b \int_0^u\beta_{F^{-1}(v)}\beta^{-1}_{F^{-1}(v)}dv} =q_0 e^{R_b u}. 
\end{eqnarray*}
Thus, the rescaled process has constant transition rate $R_b$. Three parts in the rescaled diffusion model need attentions to make it equivalent to the original model.
\begin{enumerate}
  \item The time embedding $E_r$ for rescaled process should be $E_r=E_o\circ F^{-1}$, where $E_o$ is the original time embedding.
  \item For training, the rescaled model usually sample time $u$ from $\mathcal{U}(u;0,F(T))$. On the other hand, the orginal model sample $t$ from $\mathcal{U}(t;0,T)$. Then for $u=F(t)$, by the variable substitution $s=F^{-1}(v)$,
  \begin{eqnarray*}
    \mathbb{P}(v\le u)=\int_0^t\frac{1}{T} ds=\int_0^u\frac{\partial_v F^{-1}(v)}{T}dv=\int_0^u\frac{\beta^{-1}_{F^{-1}(v)}}{T}dv.
  \end{eqnarray*}  
  Thus, the probability density function of $u$ is $\frac{1}{T\beta_{F^{-1}(u)}}$ but not $\frac{1}{F(T)}$. Since $R_t = \beta_t R_b$, the negative ELBO $\mathcal{L}^\text{CT}(\theta)$ is weighted by $\beta_t$, thereby its gradients over $\theta$ is also weighted by $\beta_t$. In summary, the time sampling of the original model is actually an importance sampling of the rescaled model. Generally, if the original model sample $t$ from density function $\phi(t)$ over $[0,T]$, then the rescale model sample $u$ from density function $\frac{\phi\circ F^{-1}(u)}{\beta_{F^{-1}(u)}}$ with weight $\beta_{F^{-1}(u)}$ for gradients.
  \item For generating, the exact Gillespie's algorithm in https://arxiv.org/pdf/2205.14987 does not change. The tau-leaping algorithm in https://arxiv.org/pdf/2205.14987, the unified discrete-time scheduler in https://arxiv.org/pdf/2402.03701, and the MCMC predictor-corrector should be rescaled accordingly.
\end{enumerate}
With out loss of generality, we always assume time-homogeneous process $\beta_t=1$ in this paper.

%\centerline{\bf \bfseries  ---------------------------------------------------------}

\bibliographystyle{unsrt}
\bibliography{reference}

\end{document}
